\documentclass[conference]{IEEEtran}
\usepackage{amsmath, amssymb}
\usepackage{graphicx}
\usepackage{cite}
\usepackage{hyperref}
\usepackage{caption}
\usepackage{float}

\title{\LARGE \bf Exploring Applications of Graph Theory in Social Network Analysis}

\author{
    \IEEEauthorblockN{Gunjan Kumari}
    \IEEEauthorblockA{
        B.Sc (Hons.) Mathematics \\
        Shaheed Rajguru College of Applied Sciences for Women \\
        University of Delhi \\
        \href{mailto:gunjankumari1096@gmail.com}{gunjankumari1096@gmail.com}
    }
}

\begin{document}
\maketitle

\begin{abstract}
Social networks are among the most complex and dynamic systems today. Graph theory provides a natural framework for modeling such networks, where users are nodes and interactions are edges. This paper explores key graph-based concepts—centrality, community detection, clustering—and applies them to real-world Facebook data. We used Python's NetworkX and Gephi for analysis and visualization, uncovering behavioral and structural insights. Applications in influence detection, recommendation, and security are also discussed.
\end{abstract}

\section{Introduction}
With billions of active users, platforms like Facebook, Instagram, and Twitter create massive networks of interactions. Understanding these relationships is vital for marketing, recommendation engines, fraud detection, and more.

Graph theory, a branch of mathematics, allows us to represent these platforms as graphs: each user becomes a node and every interaction forms an edge. Analyzing these structures enables identification of key influencers, community clusters, and hidden patterns.

This paper presents how graph theory can be applied practically to a real-world dataset and explores advanced metrics such as betweenness centrality, modularity-based clustering, and the small-world phenomenon.

\section{Graph Theory Concepts}
Let \( G = (V, E) \) be a graph with nodes \( V \) and edges \( E \).

\subsection{Degree Centrality}
Measures direct influence—number of connections.

\subsection{Closeness Centrality}
Measures average distance to all nodes. High values suggest efficiency in spreading information.

\subsection{Betweenness Centrality}
Identifies bridge nodes—those often on the shortest paths between others.

\subsection{Clustering Coefficient}
Quantifies how tightly a node's neighbors are connected.

\subsection{Modularity}
Used for community detection—measures the strength of division of a network into modules.

\section{Dataset and Tools}
We used the SNAP ego-Facebook dataset:
\begin{itemize}
  \item \textbf{Nodes:} 4,039 users
  \item \textbf{Edges:} 88,234 friendships
  \item \textbf{Graph Type:} Undirected, unweighted
\end{itemize}

\textbf{Tools:}
\begin{itemize}
  \item \textbf{Python (NetworkX):} Metric computation
  \item \textbf{Gephi:} Community visualization and layout
  \item \textbf{Matplotlib:} Distribution plots
\end{itemize}

\section{Methodology}
We constructed a graph using NetworkX and calculated centrality, clustering, and path-based metrics.

\section{Results and Discussion}
\subsection{Top Influencers}
\begin{itemize}
  \item User 107: Degree centrality = 0.028
  \item User 121: Betweenness centrality = 0.38
\end{itemize}

Gephi’s modularity algorithm detected 8 communities. Most users were clustered into small subgroups with internal cohesion.
\subsection{Degree Distribution}
We observed a power-law-like distribution typical of social networks.

\subsection{Average Path and Clustering}
\begin{itemize}
  \item Average path length: 3.5
  \item Average clustering coefficient: 0.61
\end{itemize}
\section{Applications}
\begin{itemize}
  \item \textbf{Marketing:} Target high-degree central users
  \item \textbf{Security:} Monitor bridge users (high betweenness)
\end{itemize}

\section{Conclusion and Future Work}
Graph theory provides a powerful lens to examine social networks. From centrality to clustering, each metric uncovers new behavioral or structural information.

In future work, we aim to:
\begin{itemize}
  \item Apply sentiment analysis to edge weights.
  \item Use temporal graphs for evolving networks.
\end{itemize}

\section*{References}
\begin{thebibliography}{00}
\bibitem{newman}
Newman, M. E. J. (2003). The structure and function of complex networks. \textit{SIAM Review}.

\bibitem{easley}
Easley, D., \& Kleinberg, J. (2010). \textit{Networks, Crowds, and Markets}. Cambridge University Press.
\bibitem{snap}
Jure Leskovec, SNAP Datasets. \url{https://snap.stanford.edu/data/ego-Facebook.html}

\bibitem{networkx}
NetworkX Documentation. \url{https://networkx.org/}

\bibitem{gephi}
Gephi Open Graph Platform. \url{https://gephi.org/}
\end{thebibliography}

\end{document}

  \item Integrate machine learning for predictive modeling.
  \item \textbf{Recommendation:} Use community info for content suggestion



